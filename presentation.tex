\documentclass[a4paper,11pt]{beamer}
\usepackage{etex}
\usepackage{lmodern}
\usepackage[french]{babel}
\usepackage[T1]{fontenc}
\usepackage[utf8]{inputenc}
\usepackage{pst-sigsys} 
\usepackage{amsmath,amsfonts,amssymb}
\usepackage{pstricks-add} 
\usepackage{ragged2e}
\usepackage{graphicx}   
\usepackage{ulem} 
\usepackage{wasysym}
\usepackage{matlab-prettifier}
\usepackage{hyperref}
\setbeamertemplate{navigation symbols}{}   
  
\usetheme{Darmstadt} 
\setbeamertemplate{footline}{\insertframenumber/\inserttotalframenumber}
\title{L3 - CMI017 : Signaux et Systèmes\\Présentation}
\author{Frank BULOUP - frank.buloup@univ-amu.fr} 
\institute{Aix Marseille Université\\Institut des Sciences du Mouvement}
\date{}

\setbeamertemplate{footline} 
{  
	\begin{beamercolorbox}[ht=2.5ex,dp=1.125ex,%
      leftskip=.3cm,rightskip=.3cm plus1fil]{title in head/foot}%
      {\usebeamerfont{title in head/foot}\insertshorttitle} \hfill    
      \insertframenumber / \inserttotalframenumber%
    \end{beamercolorbox}%
%     \begin{beamercolorbox}[colsep=1.5pt]{lower separation line foot}
%     \end{beamercolorbox} 
}
 
\newcounter{exampleBlockCounter}
\setcounter{exampleBlockCounter}{1} 
  
\begin{document} 


\begin{frame}[plain] 
	\titlepage 
	\center{\includegraphics[scale=0.75]{images/by-nc-sa.eps}}
	\vspace{1cm}
	
	\includegraphics[scale=0.6]{images/LogoAMU.png}\hspace*{2cm}
	\includegraphics[scale=0.2]{images/LogoCNRS.eps}\hspace*{2cm}
	\includegraphics[scale=0.1]{images/LogoISM.eps}
\end{frame} 

% \section{Présentation} 
% \subsection{Programme}

\begin{frame}[plain]
\frametitle{Objectifs de cette formation}
\center
À l'issue de cette formation, vous serez capable de :
\vspace{0.25cm}

\begin{enumerate}
  \item Décrire les concepts de signaux et systèmes
 		\begin{itemize}
 		  \item Distinguer les domaines \textit{\textbf{continu}} et
 		  \textit{\textbf{discret}}
 		  \item Expliquer le concept de système discret linéaire et invariant dans le
 		  temps (SDLIT)
 		\end{itemize}
  \item Manipuler leurs représentations temporelle et fréquentielle
  	    \begin{itemize}
  	      \item Utiliser les équations aux différences et les diagrammes blocs
  	      \item Décrire le concept de régime harmonique
 		  \item Utiliser la fonction de transfert système
 		\end{itemize}
\end{enumerate}
\begin{exampleblock}{Remarque importante}
\center{Je suis à votre disposition sur RDV fixé par mail}
\end{exampleblock}
\end{frame}


\begin{frame}[plain]
\frametitle{Programme de cette formation sur 20 heures}
\begin{enumerate}
  \item Séquence I - 2 heures
  \begin{itemize}
    \item Concepts généraux de SIGNAL et de SYSTÈME
    \item Représentations temporelles des SDLIT
  \end{itemize}
  \item Séquence II - 3 heures
  \begin{itemize}
    \item \textcolor{red}{\textbf{Rendre le travail maison sur feuille -
    Correction}}
    \item Représentation fréquentielle des signaux
    \item Projet Programmation - 1h
  \end{itemize}
  \item Séquence III - 3 heures
  \begin{itemize}
    \item \textcolor{red}{\textbf{Rendre le travail maison sur feuille -
    Correction}}
    \item Opérateur $\mathcal{R}$ et fonction de transfert en $\mathcal{R}$
    \item Systèmes non bouclé et bouclé
    \item Projet Programmation - 1h
  \end{itemize}
  \item Séquence IV - 3 heures
  \begin{itemize}
    \item \textcolor{red}{\textbf{Rendre le travail maison sur feuille -
    Correction}}
    \item Convolution
    \item La transformée en z
    \item Fonction de transfert en z
    \item Projet Programmation - 1h30 
  \end{itemize}
\end{enumerate}
\end{frame}

\begin{frame}[plain]
\frametitle{Programme de cette formation sur 20 heures}
\begin{enumerate}
  \setcounter{enumi}{4}
  \item Séquence V - 3 heures 
  \begin{itemize}
    \item \textcolor{red}{\textbf{Rendre le travail maison sur feuille -
    Correction}}
    \item Réponse en fréquence des systèmes
    \item Projet Programmation - 1h30
  \end{itemize}
  \item Séquence VI - 3 heures
  \begin{itemize}
    \item \textcolor{red}{\textbf{Rendre le travail maison sur feuille -
    Correction}}
    \item TD sur la base des CC et CF des années précédentes
    \item Projet Programmation - 1h30
  \end{itemize}
   \item Séquence VII - 3 heures
  \begin{itemize}
    \item \textcolor{red}{\textbf{CC - 1h - 1/4 de la note finale}}
    \item Correction CC - 1h 
    \item Bilan Projet Programmation - 1h
  \end{itemize}
   \item \textcolor{red}{\textbf{CF - 1h - 1/2 de la note finale}}
\end{enumerate}
\pause
\begin{alertblock}{\textbf{Attention !} Travail maison = 1/4 de la note finale}
\center{Travail maison non rendu le jour J = zéro}
\end{alertblock}
\pause
\begin{alertblock}{\textbf{Et surtout : soyez actifs !}}
\center{N'hésitez pas à poser des questions, à intervenir.}
\end{alertblock}
\end{frame}

\end{document}
